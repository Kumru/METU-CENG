\documentclass[12pt]{article}
\usepackage[utf8]{inputenc}
\usepackage{float}
\usepackage{amsmath}
\usepackage{amsfonts}


\usepackage[hmargin=3cm,vmargin=6.0cm]{geometry}
%\topmargin=0cm
\topmargin=-2cm
\addtolength{\textheight}{6.5cm}
\addtolength{\textwidth}{2.0cm}
%\setlength{\leftmargin}{-5cm}
\setlength{\oddsidemargin}{0.0cm}
\setlength{\evensidemargin}{0.0cm}



\begin{document}

\section*{Student Information } 
%Write your full name and id number between the colon and newline
%Put one empty space character after colon and before newline
Full Name : Yaşar Cahit Yıldırım \\
Id Number : 2310647 \\

% Write your answers below the section tags
\section*{Answer 1}
\subsection*{a)}
$\iota = \iota$ \\
$\iota = f \circ f^{-1}$ \\
$\iota = f \circ \iota \circ f^{-1}$ \\
$\iota = f \circ g \circ g^{-1} \circ f^{-1}$ \\
$\iota = (f \circ g) \circ (g^{-1} \circ f^{-1})$ \\
$(f \circ g)^{-1} = g^{-1} \circ f^{-1}$ \\
\subsection*{b)}
Let $x, y$ $\in A$ and $f(x) = f(y)$ and $x \neq y$\\
$g(f(x)) = g(f(y))$ so we found that $g \circ f$ is not injective which is a contradiction so $x = y$ and f is injective.\\
There can be a $k,l \in B$ that $g(k) = g(l)$ so $g$ may not be injective. 
\subsection*{c)}
Let $\forall a \in A$ and $\exists b \in B$,\\
Since $g(f(a))$ is surjective and $f(a) = b,\ g(b)$ is surjective. If $g$ is surjective for some $b \in B$ then it is surjective for all.\\
There can be a $k \in B$ that $k$ is not in range of $A$ so $f$ may not be surjective.
\section*{Answer 2}
\subsection*{a)}
If $f(x) = f(y)$ then $g(f(x)) = g(f(y))$. We also know that $g(f(x)) = x$ and $g(f(y)) = y$. So $g(f(x)) = g(f(y)) = x = y$. This concludes that if $f$ has a left inverse, then it is injective.\\
\\
Let $x = g(y)$ then $f(x) = f(g(y)) = y$ so for all $y \in B$ there exists some $x \in A$ such that $f(x) = y$. This concludes that if $f$ has a right inverse, then it is surjective.\\
\subsection*{b)}
Let $S = \mathbb{N}$ and define $f: S \rightarrow S$ by $f(x) = x + 1$. Define $g: S \rightarrow S$ by $g(y) = y - 1$ for $y > 1$. Then $g(1)$ may be defined arbitrarily and any such $g$ would be a left inverse of $f$.\\
\\
As we could prove it alike, we can also give an example to the situation for multiple right inverse. Let $A = \{a, b\}$, $B = \{c\}$, $f(a) = c$, $f(b) = c$. In this situation right inverses of $f$ are $f^{-1}(c) = a$ and $f^{-1}(c) = b$.\\
\subsection*{c)}
We proved that $f$ is both injective and surjective, this means it is bijective.\\
And,
\begin{center}
$g \circ f \circ h = g \circ f \circ h$\\
$\iota_A \circ h = g \circ \iota_B$\\
$g = h = f^{-1}$
\end{center}
\section*{Answer 3}
We can conclude $f$'s injectivity by,
\begin{center}
$f(x_1, y_1) = f(x_2, y_2)$\\
$x_1 + y_1 - 1 = x_2 + y_2 - 1$\\
$y_1 = y_2$ and $x_1 = x_2$
\end{center}
We can conclude $f$'s surjectivity by,
\begin{center}
For $a, b \in A$, let $f(x, y) = (a, b)$\\
$x = a - b + 1,\ y = b$\\
$b \leq a$ so both $x$ and $y$ are positive.
\end{center}
We can observe that $x(x-1)/2$ is the sum of all the numbers from $1$ to $x-1$.\\
$(table from\ textbook\ solutions\ 2.5.31)$\\
$g(1,1) \rightarrow$ 1\\
$g(2,1),(2,2) \rightarrow$ 2,3\\
$g(3,1),(3,2),(3,3) \rightarrow$ 4,5,6\\
.\\
.\\
.\\
We see by looking at the table that the function takes on successive values. To 
\newpage
\section*{Answer 4}
\subsection*{a)}
Cartesian product of $\mathbb{Q}^n$ is countable. It is proved using the diagonalization rule that can be used to prove rational numbers are countable too. Each polynomial equation of positive degree $n$ has at most $n$ roots, therefore these roots are countable. So algebraic numbers are unions of these polynomials that has countably many roots, which means they are countable.
\subsection*{a)}
We know that real numbers are a union of transcendental numbers and algebraic numbers. Since we proved that the algebraic number set is countable and we know that the real numbers are uncountable by the Cantor's Diagonalization Argument, assuming that transcendental numbers are countable will cause a contradiction beacuse union of countable sets are also countable. So we can conclude that transcendental numbers are uncountable.
\section*{Answer 5}
Let $c_1\ and\ c_2$ be constants that from the Big-Theta definition $c_1k \leq nlnn \leq c_2n$. We have $lnc_1 + lnk \leq lnn + ln(lnn)$ so $lnk = O(lnn)$. Let $c_3$ as that $lnk \leq c_3lnn.$ Then,
\begin{center}
$\dfrac{k}{lnk} \geq \dfrac{k}{c_3lnn} \geq \dfrac{n}{c_2c_3}$
\end{center}
so $\dfrac{k}{lnk} = \Omega(n)$. Similarly, we have $lnn + ln(lnn) \leq lnc_2 + lnk$ so $lnk = \Omega(lnn)$. Let $c_4$ as that $lnk \geq c_4lnn.$ Then,
\begin{center}
$\dfrac{k}{lnk} \leq \dfrac{k}{c_3lnn} \leq \dfrac{n}{c_1c_4}$
\end{center}
so $\dfrac{k}{lnk} = O(n)$. Using the functions obtained and symmetry, $n = \Theta(\dfrac{k}{lnk})$.
\section*{Answer 6}
\subsection*{a)}
Divisor set of 6 is \{1, 2, 3, 6\}. Their sum other than 6 is 6.\\
Divisor set of 28 is \{1, 2, 4, 7, 14, 28\}. Their sum other than 28 is 28.\\
\subsection*{b)}
Every perfect number n must have all its divisors sum up to be 2n.\\
As $2^p$-1 is prime, its divisors are the set \{1, $2^p$-1\} and their sum is $2^p$.\\
We can prove that $2^{p-1}$ has divisors sum $2^p$-1 by "k+1'th step" technique. Assume that $2^{p-1}$ has divisors sum $2^p$-1. Then $2^p$ must have divisors sum of $2^p$-1+$2^p$. It is $2^{p+1}$-1 and we have our proof.\\
So our number $2^{p-1}(2^p-1)$ has divisors sum ($2^p$-1)$2^p$. This is 2($2^p$-1)$2^{p-1}$ and our number has divisors sum ($2^p$-1)$2^{p-1}$ other than itself.
\newpage
\section*{Answer 7}
\subsection*{a)}
\begin{center}
$x \equiv c_1$   (mod $m$) \\
$x \equiv c_2$   (mod $n$) \\
\end{center}
Let $a = gcd(m, n)$.
\begin{center}
$x-c_1 \equiv 0$   (mod $m$) \\
$x-c_2 \equiv 0$   (mod $n$) \\
\end{center}
$(x-c_1)$ and $(x-c_2)$ are both multiplies of $a$. So $a$ is also a multiple of $(x-c_2) - (x-c_1) = c_1 - c_2$ so,
\begin{center}
$gcd(m, n) | c_1 - c_2$
\end{center}
Conversely, assume $gcd(m, n) | c_1 - c_2$ and let $gcd(m, n) = a$.
\begin{center}
$c_1-c_2 \equiv 0$   (mod $a$)\\
$(x-c_2) - (x-c_1) \equiv 0$   (mod $a$)\\
$(x-c_1) \equiv 0$ (mod $a$)\\
$(x-c_2) \equiv 0$ (mod $a$)\\
\end{center}
Since $m = k_1*a$ and $n = k_2*a$ where $k1, k2$ are constant,
\begin{center}
$x \equiv c_1$ (mod $m$)\\
$x \equiv c_2$ (mod $n$)
\end{center}

\subsection*{b)}
We concluded that there exist at least a solution to the system in part a. To generalize our solution above, since both $m$ and $n$ divides $lcm(m, n)$, assuming our solution is $x = a$, we can say that,
\begin{center}
$x = a + t * lcm(m, n)$ for $t \ in\ \mathbb{N}$
\end{center}
so for $t = 0$, our solution $x$ is in the interval [0, lcm(m, n)) and unique.
\end{document}
​

