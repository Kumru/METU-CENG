\documentclass[12pt]{article}
\usepackage[utf8]{inputenc}
\usepackage{float}
\usepackage{amsmath}
%\usepackage{amsfonts}


\usepackage[hmargin=3cm,vmargin=6.0cm]{geometry}
%\topmargin=0cm
\topmargin=-2cm
\addtolength{\textheight}{6.5cm}
\addtolength{\textwidth}{2.0cm}
%\setlength{\leftmargin}{-5cm}
\setlength{\oddsidemargin}{0.0cm}
\setlength{\evensidemargin}{0.0cm}



\begin{document}

\section*{Student Information } 
%Write your full name and id number between the colon and newline
%Put one empty space character after colon and before newline
Full Name :  Yaşar Cahit Yıldırım\\
Id Number :  2310647\\

% Write your answers below the section tags
\section*{Answer 1}
\subsection*{a}
In the generating function for the sequence $\{a_n\}$, where $a_n$ is the number of ways to choose $n$ candies.\\
For green candies,
\begin{center}
    $(1 + x^2 + x^4)$
\end{center}
For red candies,
\begin{center}
    $(x^4 + x^5 + x^6 + \cdot \cdot \cdot)$
\end{center}
And for blue candies,
\begin{center}
    $(x^1 + x^3 + x^5 + \cdot \cdot \cdot)$
\end{center}
From these, we can get generating function
\begin{center}
    $(1 + x^2 + x^4)(x^4 + x^5 + x^6 + \cdot \cdot \cdot)(x^1 + x^3 + x^5 + \cdot \cdot \cdot)$
\end{center}
We will calculate the term $x^{10}$s coefficient for the solution. So the answer is;
\begin{center}
    $[x^{10}] (1 + x^2 + x^4)(x^4 + x^5 + x^6 + \cdot \cdot \cdot)(x^1 + x^3 + x^5 + \cdot \cdot \cdot) = 6$
\end{center}
\subsection*{b}
Recreating the functions above to make them have at most 5 as an exponent.\\
For green candies it does not change,
\begin{center}
    $(1 + x^2 + x^4)$
\end{center}
For red candies,
\begin{center}
    $(x^4 + x^5)$
\end{center}
And for blue candies,
\begin{center}
    $(x^1 + x^3 + x^5)$
\end{center}
From these, we can get generating function
\begin{center}
    $(1 + x^2 + x^4)(x^4 + x^5)(x^1 + x^3 + x^5)$
\end{center}
We will, again,  calculate the term $x^{10}$s coefficient for the solution. So the answer is;
\begin{center}
    $[x^{10}] (1 + x^2 + x^4)(x^4 + x^5)(x^1 + x^3 + x^5) = 3$
\end{center}
\subsection*{c}
\begin{center}
    $F(x) = 7x^2 \dfrac{1}{(1-2x)} \dfrac{1}{(1+3x)} = 7x^2\bigg(\dfrac{A}{1-2x}+\dfrac{B}{1+3x}\bigg)=\dfrac{7x^2}{5}\bigg(\dfrac{2}{1-2x}+\dfrac{3}{1-(-3x)}\bigg)$
\end{center}
We know that,
    $$\dfrac{1}{1-x} = \sum_{k\geq0}^{\infty} x^k$$
So we will replace $x$ in $1-x$ with $2x$ and $-3x$ for the last part of the function, respectively.
    $$\dfrac{2}{1-2x} = 2\sum_{k\geq0}^{\infty} 2^k x^k = \sum_{k\geq0}^{\infty} 2^{k+1} x^k$$
and
    $$\dfrac{3}{1-(-3x)} = 3\sum_{k\geq0}^{\infty} (-3)^k x^k = -\sum_{k\geq0}^{\infty} (-3)^{k+1} x^k$$
We now conclude the function $F(x)$ corresponds to series $\dfrac{7}{5} \sum_{k\geq2}^{\infty} (2^{k-1}-((-3)^{k-1})) x^{k}$, after right shifting. So the rule for the sequence is:
\begin{center}
    $a_0 = a_1 = 0$ and $a_k = \dfrac{7}{5}(2^{k-1}-(-3)^{k-1})$ where $k>1$.
\end{center}
\subsection*{d}
We will set $a_0 = 1$ where it provides $a_1 = 8a_0 + 10^0 = 9$ to ease our job. Let $G(x) = \sum_{n=0}^{\infty}a_n x^n$.
\begin{align*}
    \sum_{n=1}^{\infty} a_n x^n = G(x) - a_0 = G(x) - 1 &= \sum_{n=1}^{\infty}(8a_{n-1}x^n + 10^{n-1}x^n) \\
    &= \sum_{n=0}^{\infty}8a_n x^{n+1} + \sum_{n=0}^{\infty} 10^n x^{n+1} \\
    &= 8x\sum_{n=0}^{\infty}a_n x^n + x\sum_{n=0}^{\infty} 10^n x^n \\
    &= 8xG(x) + \dfrac{x}{1-10x}
\end{align*}
\begin{center}
    \begin{align*}
        G(x) = \dfrac{1-9x}{(1-8x)(1-10x)} = \dfrac{A}{1-8x} + \dfrac{B}{1-10x} &= \dfrac{1/2}{1-8x} + \dfrac{1/2}{1-10x} \\
        &= \dfrac{1}{2}\bigg(\dfrac{1}{1-8x} + \dfrac{1}{1-10x}\bigg)
    \end{align*}
    \bigskip
    Solution is $G(x) = \dfrac{1}{2}\big(8^n + 10^n\big)$.
\end{center}
\newpage
\section*{Answer 2}
\subsection*{a}
If $m | k$, then $k=am$ where $a$ is a positive integer.\\
Let $l\in A_k$ for arbitrary $l$. $l = bk$ where $b$ is a positive integer and since $k=am$, $l=bam=(ba)m=cm$ where $c$ is a positive integer. So we conclude that $m|l$ holds.
\begin{center}
    $A_k \subseteq A_m$
\end{center}
\subsection*{b}
We know that a number $n$ is prime if and only if it has no prime divisor $d$ where $d\leq\sqrt{n}$. So every composite number $c\leq n$ has prime factors less than or equal to $\sqrt{n}$ and can be constructed using these.\\
Second equation provides this and first equation is defined as all composite numbers which are less than or equal to than $n$. They are equal.
\subsection*{c}
In the interval $[m, ..., n]$; for every $m$ number, there is a number that $m$ divides. We can show that as $\bigg\lfloor \dfrac{n}{m} \bigg\rfloor$ and when we exclude $m$ itself we get $\bigg\lfloor \dfrac{n}{m} \bigg\rfloor -1$. So,
\begin{center}
    $|A_m| = \bigg\lfloor \dfrac{n}{m} \bigg\rfloor -1$
\end{center}
\subsection*{d}
They are exactly the same sets except $A_{ab}$ does not include itself so it is $ab$ if $ab\leq n$.
\subsection*{e}
Let $t$ be product of all $p\in P$.
\begin{displaymath}
    \bigg| \bigcap\limits_{p \in P} A_p \bigg| = \bigg\lfloor \dfrac{n}{t} \bigg\rfloor
\end{displaymath}
\subsection*{f}
\begin{center}
    $|C_{45}| = |A_2| + |A_3| + |A_5| - |A_2\cap A_3| - |A_2\cap A_5| - |A_3\cap A_5| + |A_2\cap A_3\cap A_5|$
\end{center}
\subsection*{g}
\begin{center}
    $45 - |C_{45}| - |\{1\}| = |P_{45}| = 45 - 21 - 14 - 8 + 7 + 4 + 3 - 1 - 1 = 14$
\end{center}
\section*{Answer 3}
\subsection*{a}
Knowing $\big((a, b) \ll (c, d) \in R\big)$ and $\big((c, d) \ll (e, f) \in R\big)$, if we can get $(a, b) \ll (e, f)$ then we can say that $\ll$ is transitive.
\begin{center}
    $a<c \land (a=c \land b\leq d)$ is true.\\
    $c<e \land (c=e \land d\leq f)$ is true.
\end{center}
From here we can deduct that,
\begin{center}
    $a<e \land (a=e \land b\leq f)$ is true too.
\end{center}
So $\ll$ is transitive.
\subsection*{b}
A relation is called an equivalence relation if it is reflexive, symmetric, and transitive. So showing these three properties hold for $\propto$ will be enough to say it is an equivalence relation.\\
\begin{center}
    $f(x) = f(x)$ is obvious and trivial.\\
    \bigskip
    If $f(x) = g(x)$ for all $x\geq k$ where $k\in {\rm I\!R}$, then $g(x) = f(x)$, too.\\
    \bigskip
    If $f(x) = g(x)$ and $g(y) = h(y)$ for all $x\geq k$ and $y\geq l$ where $k,l\in{\rm I\!R}$, then $f(z) = h(z)$ for all $z\geq m$ where $m\in{\rm I\!R}$.
\end{center}
So we conclude that $\propto$ is an equivalence relation.

\end{document}

​

